\documentclass[12pt,a4paper]{report}
\usepackage[utf8]{inputenc}
\usepackage[german]{babel}
\usepackage{amsmath}
\usepackage{amsfonts}
\usepackage{amssymb}
\usepackage{graphicx}
\author{Seitz Andre}
\title{ML Worksheet1}
\begin{document}
\section{Problem 1}
For calculation of eigenvalues and eigenvectors of the given matrix A
\begin{equation}
det(A-\lambda I) = 0
\end{equation}
was used. First the eigenvalues $\lambda$ were calculated.
\begin{eqnarray}
det ( A - \lambda I ) = 0 \\
det ( \left(
\begin{array}{ccc}
2 & -1 & 0 \\
-1 & 2 & -1 \\
0 & -1 & 2
\end{array}
\right) - \lambda I ) = 0 \\
\left|
\begin{array}{ccc}
2-\lambda & -1 & 0 \\
-1 & 2-\lambda & -1 \\
0 & -1 & 2-\lambda 
\end{array}
\right| = 0 \\
(2-\lambda) \cdot \left|
\begin{array}{cc}
2-\lambda & -1 \\
-1 & 2-\lambda 
\end{array} \right|
- (-1) \cdot \left|
\begin{array}{cc}
-1 & 0 \\
-1 & 2-\lambda
\end{array} \right| = 0 \\
(2-\lambda)\cdot((2-\lambda)^2-1)+(-1)\cdot(2-\lambda)-0\cdot(-1) = 0\\
(2-\lambda)\cdot((2-\lambda)^2-1)+(-1)\cdot(2-\lambda) = 0
\end{eqnarray}
By division through $(2- \lambda)$ the first $\lambda$ can be derived as $\lambda_1 = 2$.
\begin{eqnarray}
((2-\lambda)^2-1)+(-1) = 0 \\
(2-\lambda)^2-2 = 0 \\
\lambda^2-4\lambda+4-2 = 0 \\
\lambda^2-4\lambda+2 = 0 \\
\lambda_{2,3} = 2\pm \sqrt{2}
\end{eqnarray}
The eigenvectors of the corresponding eigenvalues can now be derived by using $(A - \lambda I)X = 0$. \\
For $\lambda = 2$
\begin{eqnarray}
\left(
\begin{array}{ccc}
0 & -1 & 0 \\
-1 & 0 & -1 \\
0 & -1 & 0
\end{array} \right) \cdot X = 0
\end{eqnarray}
resolves to the following system of linear equations:
\begin{eqnarray}
-x_2 = 0 \\
-x_1 -x_3 = 0 \\
-x_2 = 0
\end{eqnarray}
This system can be solved to
\begin{eqnarray}
x_2 = 0 \\
x_1 = -x_3
\end{eqnarray}
and leads to the first eigenvectors
\begin{equation}
\left( \begin{array}{c}
x_1 \\ 0 \\ -x_1
\end{array} \right)
\end{equation}

\end{document}