\documentclass{article}
\usepackage{ml1_homework_template}

\usepackage{amssymb}

\usepackage{fancyvrb}
\usepackage{color}


\makeatletter
\def\PY@reset{\let\PY@it=\relax \let\PY@bf=\relax%
    \let\PY@ul=\relax \let\PY@tc=\relax%
    \let\PY@bc=\relax \let\PY@ff=\relax}
\def\PY@tok#1{\csname PY@tok@#1\endcsname}
\def\PY@toks#1+{\ifx\relax#1\empty\else%
    \PY@tok{#1}\expandafter\PY@toks\fi}
\def\PY@do#1{\PY@bc{\PY@tc{\PY@ul{%
    \PY@it{\PY@bf{\PY@ff{#1}}}}}}}
\def\PY#1#2{\PY@reset\PY@toks#1+\relax+\PY@do{#2}}

\expandafter\def\csname PY@tok@gd\endcsname{\def\PY@tc##1{\textcolor[rgb]{0.63,0.00,0.00}{##1}}}
\expandafter\def\csname PY@tok@gu\endcsname{\let\PY@bf=\textbf\def\PY@tc##1{\textcolor[rgb]{0.50,0.00,0.50}{##1}}}
\expandafter\def\csname PY@tok@gt\endcsname{\def\PY@tc##1{\textcolor[rgb]{0.00,0.27,0.87}{##1}}}
\expandafter\def\csname PY@tok@gs\endcsname{\let\PY@bf=\textbf}
\expandafter\def\csname PY@tok@gr\endcsname{\def\PY@tc##1{\textcolor[rgb]{1.00,0.00,0.00}{##1}}}
\expandafter\def\csname PY@tok@cm\endcsname{\let\PY@it=\textit\def\PY@tc##1{\textcolor[rgb]{0.25,0.50,0.50}{##1}}}
\expandafter\def\csname PY@tok@vg\endcsname{\def\PY@tc##1{\textcolor[rgb]{0.10,0.09,0.49}{##1}}}
\expandafter\def\csname PY@tok@m\endcsname{\def\PY@tc##1{\textcolor[rgb]{0.40,0.40,0.40}{##1}}}
\expandafter\def\csname PY@tok@mh\endcsname{\def\PY@tc##1{\textcolor[rgb]{0.40,0.40,0.40}{##1}}}
\expandafter\def\csname PY@tok@go\endcsname{\def\PY@tc##1{\textcolor[rgb]{0.53,0.53,0.53}{##1}}}
\expandafter\def\csname PY@tok@ge\endcsname{\let\PY@it=\textit}
\expandafter\def\csname PY@tok@vc\endcsname{\def\PY@tc##1{\textcolor[rgb]{0.10,0.09,0.49}{##1}}}
\expandafter\def\csname PY@tok@il\endcsname{\def\PY@tc##1{\textcolor[rgb]{0.40,0.40,0.40}{##1}}}
\expandafter\def\csname PY@tok@cs\endcsname{\let\PY@it=\textit\def\PY@tc##1{\textcolor[rgb]{0.25,0.50,0.50}{##1}}}
\expandafter\def\csname PY@tok@cp\endcsname{\def\PY@tc##1{\textcolor[rgb]{0.74,0.48,0.00}{##1}}}
\expandafter\def\csname PY@tok@gi\endcsname{\def\PY@tc##1{\textcolor[rgb]{0.00,0.63,0.00}{##1}}}
\expandafter\def\csname PY@tok@gh\endcsname{\let\PY@bf=\textbf\def\PY@tc##1{\textcolor[rgb]{0.00,0.00,0.50}{##1}}}
\expandafter\def\csname PY@tok@ni\endcsname{\let\PY@bf=\textbf\def\PY@tc##1{\textcolor[rgb]{0.60,0.60,0.60}{##1}}}
\expandafter\def\csname PY@tok@nl\endcsname{\def\PY@tc##1{\textcolor[rgb]{0.63,0.63,0.00}{##1}}}
\expandafter\def\csname PY@tok@nn\endcsname{\let\PY@bf=\textbf\def\PY@tc##1{\textcolor[rgb]{0.00,0.00,1.00}{##1}}}
\expandafter\def\csname PY@tok@no\endcsname{\def\PY@tc##1{\textcolor[rgb]{0.53,0.00,0.00}{##1}}}
\expandafter\def\csname PY@tok@na\endcsname{\def\PY@tc##1{\textcolor[rgb]{0.49,0.56,0.16}{##1}}}
\expandafter\def\csname PY@tok@nb\endcsname{\def\PY@tc##1{\textcolor[rgb]{0.00,0.50,0.00}{##1}}}
\expandafter\def\csname PY@tok@nc\endcsname{\let\PY@bf=\textbf\def\PY@tc##1{\textcolor[rgb]{0.00,0.00,1.00}{##1}}}
\expandafter\def\csname PY@tok@nd\endcsname{\def\PY@tc##1{\textcolor[rgb]{0.67,0.13,1.00}{##1}}}
\expandafter\def\csname PY@tok@ne\endcsname{\let\PY@bf=\textbf\def\PY@tc##1{\textcolor[rgb]{0.82,0.25,0.23}{##1}}}
\expandafter\def\csname PY@tok@nf\endcsname{\def\PY@tc##1{\textcolor[rgb]{0.00,0.00,1.00}{##1}}}
\expandafter\def\csname PY@tok@si\endcsname{\let\PY@bf=\textbf\def\PY@tc##1{\textcolor[rgb]{0.73,0.40,0.53}{##1}}}
\expandafter\def\csname PY@tok@s2\endcsname{\def\PY@tc##1{\textcolor[rgb]{0.73,0.13,0.13}{##1}}}
\expandafter\def\csname PY@tok@vi\endcsname{\def\PY@tc##1{\textcolor[rgb]{0.10,0.09,0.49}{##1}}}
\expandafter\def\csname PY@tok@nt\endcsname{\let\PY@bf=\textbf\def\PY@tc##1{\textcolor[rgb]{0.00,0.50,0.00}{##1}}}
\expandafter\def\csname PY@tok@nv\endcsname{\def\PY@tc##1{\textcolor[rgb]{0.10,0.09,0.49}{##1}}}
\expandafter\def\csname PY@tok@s1\endcsname{\def\PY@tc##1{\textcolor[rgb]{0.73,0.13,0.13}{##1}}}
\expandafter\def\csname PY@tok@sh\endcsname{\def\PY@tc##1{\textcolor[rgb]{0.73,0.13,0.13}{##1}}}
\expandafter\def\csname PY@tok@sc\endcsname{\def\PY@tc##1{\textcolor[rgb]{0.73,0.13,0.13}{##1}}}
\expandafter\def\csname PY@tok@sx\endcsname{\def\PY@tc##1{\textcolor[rgb]{0.00,0.50,0.00}{##1}}}
\expandafter\def\csname PY@tok@bp\endcsname{\def\PY@tc##1{\textcolor[rgb]{0.00,0.50,0.00}{##1}}}
\expandafter\def\csname PY@tok@c1\endcsname{\let\PY@it=\textit\def\PY@tc##1{\textcolor[rgb]{0.25,0.50,0.50}{##1}}}
\expandafter\def\csname PY@tok@kc\endcsname{\let\PY@bf=\textbf\def\PY@tc##1{\textcolor[rgb]{0.00,0.50,0.00}{##1}}}
\expandafter\def\csname PY@tok@c\endcsname{\let\PY@it=\textit\def\PY@tc##1{\textcolor[rgb]{0.25,0.50,0.50}{##1}}}
\expandafter\def\csname PY@tok@mf\endcsname{\def\PY@tc##1{\textcolor[rgb]{0.40,0.40,0.40}{##1}}}
\expandafter\def\csname PY@tok@err\endcsname{\def\PY@bc##1{\setlength{\fboxsep}{0pt}\fcolorbox[rgb]{1.00,0.00,0.00}{1,1,1}{\strut ##1}}}
\expandafter\def\csname PY@tok@kd\endcsname{\let\PY@bf=\textbf\def\PY@tc##1{\textcolor[rgb]{0.00,0.50,0.00}{##1}}}
\expandafter\def\csname PY@tok@ss\endcsname{\def\PY@tc##1{\textcolor[rgb]{0.10,0.09,0.49}{##1}}}
\expandafter\def\csname PY@tok@sr\endcsname{\def\PY@tc##1{\textcolor[rgb]{0.73,0.40,0.53}{##1}}}
\expandafter\def\csname PY@tok@mo\endcsname{\def\PY@tc##1{\textcolor[rgb]{0.40,0.40,0.40}{##1}}}
\expandafter\def\csname PY@tok@kn\endcsname{\let\PY@bf=\textbf\def\PY@tc##1{\textcolor[rgb]{0.00,0.50,0.00}{##1}}}
\expandafter\def\csname PY@tok@mi\endcsname{\def\PY@tc##1{\textcolor[rgb]{0.40,0.40,0.40}{##1}}}
\expandafter\def\csname PY@tok@gp\endcsname{\let\PY@bf=\textbf\def\PY@tc##1{\textcolor[rgb]{0.00,0.00,0.50}{##1}}}
\expandafter\def\csname PY@tok@o\endcsname{\def\PY@tc##1{\textcolor[rgb]{0.40,0.40,0.40}{##1}}}
\expandafter\def\csname PY@tok@kr\endcsname{\let\PY@bf=\textbf\def\PY@tc##1{\textcolor[rgb]{0.00,0.50,0.00}{##1}}}
\expandafter\def\csname PY@tok@s\endcsname{\def\PY@tc##1{\textcolor[rgb]{0.73,0.13,0.13}{##1}}}
\expandafter\def\csname PY@tok@kp\endcsname{\def\PY@tc##1{\textcolor[rgb]{0.00,0.50,0.00}{##1}}}
\expandafter\def\csname PY@tok@w\endcsname{\def\PY@tc##1{\textcolor[rgb]{0.73,0.73,0.73}{##1}}}
\expandafter\def\csname PY@tok@kt\endcsname{\def\PY@tc##1{\textcolor[rgb]{0.69,0.00,0.25}{##1}}}
\expandafter\def\csname PY@tok@ow\endcsname{\let\PY@bf=\textbf\def\PY@tc##1{\textcolor[rgb]{0.67,0.13,1.00}{##1}}}
\expandafter\def\csname PY@tok@sb\endcsname{\def\PY@tc##1{\textcolor[rgb]{0.73,0.13,0.13}{##1}}}
\expandafter\def\csname PY@tok@k\endcsname{\let\PY@bf=\textbf\def\PY@tc##1{\textcolor[rgb]{0.00,0.50,0.00}{##1}}}
\expandafter\def\csname PY@tok@se\endcsname{\let\PY@bf=\textbf\def\PY@tc##1{\textcolor[rgb]{0.73,0.40,0.13}{##1}}}
\expandafter\def\csname PY@tok@sd\endcsname{\let\PY@it=\textit\def\PY@tc##1{\textcolor[rgb]{0.73,0.13,0.13}{##1}}}

\def\PYZbs{\char`\\}
\def\PYZus{\char`\_}
\def\PYZob{\char`\{}
\def\PYZcb{\char`\}}
\def\PYZca{\char`\^}
\def\PYZam{\char`\&}
\def\PYZlt{\char`\<}
\def\PYZgt{\char`\>}
\def\PYZsh{\char`\#}
\def\PYZpc{\char`\%}
\def\PYZdl{\char`\$}
\def\PYZhy{\char`\-}
\def\PYZsq{\char`\'}
\def\PYZdq{\char`\"}
\def\PYZti{\char`\~}
% for compatibility with earlier versions
\def\PYZat{@}
\def\PYZlb{[}
\def\PYZrb{]}
\makeatother

% please submit the corresponding pdf by email to
% homework@class,brml.org, and write "homework sheet xx" in the 
% title.  No more, no less!  (Instead of xx, however,
% put the decimal number of the homework sheet.)

% Please update the following line, only change XX to the homework
% sheet number
\title{homework sheet 01}


\author{
\name{Andre Seitz}\\
\imat{3622870}\\
\email{andre.seitz@mytum.de}
\And
\name{Linda Leidig} \\
\imat{3608416}\\
\email{linda.leidig@tum.de}
}

% The \author macro works with any number of authors. There are two commands
% used to separate the names and addresses of multiple authors: \And and \AND.
%
% Using \And between authors leaves it to \LaTeX{} to determine where to break
% the lines. Using \AND forces a linebreak at that point. So, if \LaTeX{}
% puts 3 of 4 authors names on the first line, and the last on the second
% line, try using \AND instead of \And before the third author name.



\begin{document}
\maketitle

\section{Assignment}		%1
For the calculation of eigenvalues and eigenvectors of the given matrix A
\begin{eqnarray}
A x &=& \lambda x \label{definition}
\\
(A-\lambda I)x &=& 0\\
det(A-\lambda I) &=& 0
\end{eqnarray}
%\begin{equation}
%det(A-\lambda I) = 0
%\end{equation}

was used with $x$ and $\lambda$ being the eigenvectors and the eigenvalues of A, respectively. First the eigenvalues $\lambda$ were calculated.

\begin{eqnarray}
det ( A - \lambda I ) = 0 \\
det ( \left(
\begin{array}{ccc}
2 & -1 & 0 \\
-1 & 2 & -1 \\
0 & -1 & 2
\end{array}
\right) - \lambda I ) = 0 \\
\left|
\begin{array}{ccc}
2-\lambda & -1 & 0 \\
-1 & 2-\lambda & -1 \\
0 & -1 & 2-\lambda 
\end{array}
\right| = 0 \\
(2-\lambda) \cdot \left|
\begin{array}{cc}
2-\lambda & -1 \\
-1 & 2-\lambda 
\end{array} \right|
- (-1) \cdot \left|
\begin{array}{cc}
-1 & 0 \\
-1 & 2-\lambda
\end{array} \right| = 0 \\
(2-\lambda)\cdot((2-\lambda)^2-1)+(-1)\cdot(2-\lambda)-0\cdot(-1) = 0\\
(2-\lambda)\cdot((2-\lambda)^2-1)+(-1)\cdot(2-\lambda) = 0
\end{eqnarray}

By division through $(2- \lambda)$ the first $\lambda$ can be derived as $\lambda_1 = 2$.
\begin{eqnarray}
((2-\lambda)^2-1)+(-1) = 0 \\
(2-\lambda)^2-2 = 0 \\
\lambda^2-4\lambda+4-2 = 0 \\
\lambda^2-4\lambda+2 = 0 \\
\lambda_{2,3} = 2\pm \sqrt{2}
\end{eqnarray}
The eigenvectors of the corresponding eigenvalues can now be derived by using $(A - \lambda I)e = 0$.

\
For $\lambda_1 = 2$:
\begin{eqnarray}
\left(
\begin{array}{ccc}
0 & -1 & 0 \\
-1 & 0 & -1 \\
0 & -1 & 0
\end{array} \right) \cdot x = 0
\end{eqnarray}
resolves to the following system of linear equations:
\begin{eqnarray}
-x_2 = 0 \\
-x_1 -x_3 = 0 \\
-x_2 = 0
\end{eqnarray}
This system can be solved to
\begin{eqnarray}
x_2 &=& 0 \\
x_1 &=& -x_3
\end{eqnarray}

and leads to the eigenvectors
\begin{equation}
\left( \begin{array}{c}
x_1 \\ 0 \\ -x_1
\end{array} \right)
\end{equation}

Choosing $x_1 = 1$ results in
\begin{equation}
\left( \begin{array}{c}
1 \\ 0 \\ -1
\end{array} \right)
\end{equation}
as an eigenvector of $A$ to the eigenvalue $\lambda_1 = 2$.

For $\lambda_2 = 2 - \sqrt{2}$:
\begin{eqnarray}
\left(
\begin{array}{ccc}
\sqrt{2} & -1 & 0 \\
-1 & \sqrt{2} & -1 \\
0 & -1 & \sqrt{2}
\end{array} \right) \cdot x = 0
\end{eqnarray}
resolves to the following system of linear equations:
\begin{eqnarray}
\sqrt{2}x_1-x_2 = 0 \\
-x_1 +\sqrt{2} x_2 -x_3 = 0 \\
-x_2 + \sqrt{2} x_3 = 0
\end{eqnarray}
This results in 
\begin{eqnarray}
x_1 &=& x_3 \\
x_2 &=& \frac{2x_1}{\sqrt{2}}
\end{eqnarray}

and leads to the eigenvectors
\begin{equation}
\left( \begin{array}{c}
x_1 \\ \sqrt{2} x_1 \\ x_1
\end{array} \right)
\end{equation}

Choosing $x_1 = 1$ results in
\begin{equation}
\left( \begin{array}{c}
1 \\ \sqrt{2} \\ 1
\end{array} \right)
\end{equation}
as an eigenvector of $A$ to the eigenvalue $\lambda_2 = 2- \sqrt{2}$.

For $\lambda_3 = 2 + \sqrt{2}$:
\begin{eqnarray}
\left(
\begin{array}{ccc}
-\sqrt{2} & -1 & 0 \\
-1 & -\sqrt{2} & -1 \\
0 & -1 & -\sqrt{2}
\end{array} \right) \cdot x = 0
\end{eqnarray}
resolves to the following system of linear equations:
\begin{eqnarray}
-\sqrt{2}x_1-x_2 = 0 \\
-x_1 -\sqrt{2} x_2 -x_3 = 0 \\
-x_2 - \sqrt{2} x_3 = 0
\end{eqnarray}
This results in 
\begin{eqnarray}
x_1 &=& x_3 \\
x_2 &=& \frac{2x_1}{-\sqrt{2}}
\end{eqnarray}

and leads to the eigenvectors
\begin{equation}
\left( \begin{array}{c}
x_1 \\ -\sqrt{2} x_1 \\ x_1
\end{array} \right)
\end{equation}

Choosing $x_1 = 1$ results in
\begin{equation}
\left( \begin{array}{c}
1 \\ -\sqrt{2} \\ 1
\end{array} \right)
\end{equation}
as an eigenvector of $A$ to the eigenvalue $\lambda_3 = 2 + \sqrt{2}$.



The python code, solving the same problem:

\begin{Verbatim}[commandchars=\\\{\}]
\PY{k+kn}{import} \PY{n+nn}{numpy} \PY{k+kn}{as} \PY{n+nn}{np}
\PY{k+kn}{from} \PY{n+nn}{numpy} \PY{k+kn}{import} \PY{n}{linalg}

\PY{n}{A} \PY{o}{=} \PY{n}{np}\PY{o}{.}\PY{n}{array}\PY{p}{(}\PY{p}{[}\PY{p}{(}\PY{l+m+mi}{2}\PY{p}{,} \PY{o}{\PYZhy{}}\PY{l+m+mi}{1}\PY{p}{,} \PY{l+m+mi}{0}\PY{p}{)}\PY{p}{,} \PY{p}{(}\PY{o}{\PYZhy{}}\PY{l+m+mi}{1}\PY{p}{,} \PY{l+m+mi}{2}\PY{p}{,} \PY{o}{\PYZhy{}}\PY{l+m+mi}{1}\PY{p}{)}\PY{p}{,} \PY{p}{(}\PY{l+m+mi}{0}\PY{p}{,} \PY{o}{\PYZhy{}}\PY{l+m+mi}{1}\PY{p}{,} \PY{l+m+mi}{2}\PY{p}{)}\PY{p}{]}\PY{p}{)}

\PY{n}{w}\PY{p}{,} \PY{n}{v} \PY{o}{=} \PY{n}{linalg}\PY{o}{.}\PY{n}{eig}\PY{p}{(}\PY{n}{A}\PY{p}{)}

\PY{k}{print}\PY{p}{(}\PY{l+s}{\PYZdq{}}\PY{l+s}{The eigenvalues are:}\PY{l+s}{\PYZdq{}}\PY{p}{)}
\PY{k}{print} \PY{n}{w}
\PY{k}{print}\PY{p}{(}\PY{l+s}{\PYZdq{}}\PY{l+s}{and the eigenvectors are:}\PY{l+s}{\PYZdq{}}\PY{p}{)}
\PY{k}{print} \PY{n}{v}
\end{Verbatim}



\section{Assignment}			%2
Let $B \in \mathbb{R}^{n \times n}$ be a matrix with $n$ linearly independent eigenvectors $x_1, x_2, ..., x_n$. $U$ is a matrix having these eigenvectors as columns and $D$ is a diagonal matrix with the corresponding eigenvalues $\lambda_i$ in the diagonal.

To show:
\begin{equation}
B = UDU^{-1}
\end{equation}

Using the rules of matrix multiplication the following transformations can be performed:
\begin{eqnarray}
B &=& UDU^{-1}\\
BU &=& UDU^{-1}U\\
BU &=& UD
\end{eqnarray}
Let $j \in {1, 2, ..., n}$.

The multiplication of $B$ with $U$ correlates with the multiplication of $B$ with each column of $U$. The $n$ resulting values written as matrix come up with the result of $BU$. In formulas:
 \begin{eqnarray}
 BU = (BU_{.j})_{j=1,...,n} = (B x_j)_{j=1,...,n}
 \label{left}
 \end{eqnarray}
 
Multiplying $U$ and $D$ results in a matrix where column $j$ corresponds to the product of the eigenvector $e_j$ i.e. column $j$ of $U$ and the eigenvalue $\lambda _j$. In formulas:
\begin{eqnarray}
(UD_{.j})_{j = 1,...,n} = (U_{.j}\lambda_j)_{j = 1,...,n} = (x_j \lambda_j)_{j=1,...,n}
\label{right}
\end{eqnarray}

With \ref{left} and \ref{right} we can say:
\begin{eqnarray}
BU &=& UD\\
B x_j &=& \lambda_j x_j \;\;\; \forall j \in {1,2,...,n}
\end{eqnarray}
which comes up with the definition of eigenvalues and eigenvectors (see \ref{definition}).


\section{Assignment}		%3
(1):

Let $A \in \mathbb{R}^n\times n$  
be real and symmetric. Next assume 
\begin{equation}
\label{linc}
\lambda \in \mathbb{C}
\end{equation}
 being an eigenvalue and x its corresponding eigenvector with
 \begin{equation}
 \label{xinc}
 x \in \mathbb{C}
 \end{equation}
 The eigenvalue and eigenvector is defined as
 \begin{equation}
 \label{nonconj}
 Ax=\lambda x
 \end{equation}
 and from \ref{linc} and \ref{xinc} follows, that also
 \begin{eqnarray}
 Ax \in \mathbb{C} \\
 \lambda x \in \mathbb{C}
 \end{eqnarray}
 Let further $\bar{x}$ and $\bar{\lambda}$ be the complex conjugates of x and y respectively.
 Because of
 \begin{eqnarray}
 (a-bi)(c-di) = \\
 ac - adi - bci + bdi^2 = \\
 ac - bd - (bc + ad)i = \\
 \bar{ac - bd + (bc+ad)i} = \\
 \bar{ac +bci + adi + bdi^2} = \\
 \bar{(a+bi)(c+di)}
 \end{eqnarray}
 the following equation is alos true
 \begin{equation}
 \label{conj}
 A\bar{x} = \bar{\lambda}\bar{x}
 \end{equation}
 A left-sided multiplication of \ref{conj} and \ref{nonconj} with the transposed vector x or its conjugate is valid. So the following can be written:
 \begin{equation}
 \label{one}
 \bar{x^T}Ax = \bar{x^T} \lambda x
 \end{equation}
 \begin{equation}
 \label{two}
  x^TA\bar{x} = x^T \bar{\lambda} \bar{x}
 \end{equation}
 By subtracting \ref{two} from \ref{one}, the following equation can be stated:
 \begin{equation}
 \bar{x^T} A x - x^T A \bar{x} = \bar{x^T} \lambda x - x^T \bar{\lambda} \bar{x} = (\lambda - \bar{\lambda}\bar{x^T}x)
 \end{equation}
 As the left side of the equation equals zero, due to the symmetry property of A, also the right side has to equal zero.
 As $\bar{x^T}x$ cannot be zero, due to x and $\bar{x^T}$ not being the nullvector, $\lambda-\bar{\lambda}$ has to be zero:
 \begin{equation}
 \lambda - \bar{\lambda} = 0 \Rightarrow \lambda = \bar{\lambda} \Rightarrow \lambda \in \mathbb{R}
 \end{equation}
 
 
(2):

 Assume
$\lambda_i \neq \lambda_j$
The definition of eigenvalues and eigenvectors is
\begin{eqnarray}
Ax_i = x_i\lambda_i \\
Ax_j = x_j\lambda_j
\end{eqnarray}
By left side multiplication with $x_j$ and $x_i$ one gets
\begin{eqnarray}
x_jAx_i = x_jx_i\lambda_i \\
x_iAx_j = x_ix_j\lambda_j
\end{eqnarray}
 Subtracting the equations leads to
 \begin{equation}
 x_jAx_i - x_iAx_j = x_jx_i\lambda_i - x_ix_j\lambda_j
\end{equation}  
Because of the symmetry of A, the left side of the equation solves to zero.
\begin{equation}
 0 = x_jx_i\lambda_i - x_ix_j\lambda_j = (\lambda_i-\lambda_j)(x_ix_j)
\end{equation}
As $\lambda_i-\lambda_j$ cannot be zero, due to the first assumption ($\lambda_i \neq \lambda_j$), the right part of the term ($x_ix_J$) has to be zero. By definition, this means, that $x_i$ and $x_j$ are orthogonal.





\section{Assignment}		%4
$B \in \mathbb{R}^{n \times n}$ is a real symmetric matrix with distinct eigenvalues $\lambda_1, \lambda_2, \lambda_n$.

To show:
\begin{eqnarray}
(1)\;\;\; |B|=\prod_i \lambda_i\\
(2) \;\;\; tr(B) = \sum_i \lambda_i
\end{eqnarray}

(1):

Since $B$ is a real symmetric matrix with distinct eigenvalues, its eigenvectors are orthogonal (see Assignment 3). Therefore, the eigenvectors are linearly independent, which means that it is possible to diagonalize the $B$.

In addition, we know that the determinant of a diagonal matrix is the same as the multiplication of all values on the diagonal.

Let $U$ and $D$ be matrices as defined in Assignment 2.
\begin{eqnarray}
B &=& UDU^{-1}\\
|B| &=& |UDU^{-1}|\\
|B| &=& |U||D||U^{-1}|\\
|B| &=& |U||D||U|^{-1}\\
|B| &=& \frac{|U|}{|U|}|D|\\
|B| &=& |D|\\
|B| &=& \prod_i \lambda_i
\end{eqnarray}

(2):

We use the same facts described in (1):
\begin{eqnarray}
B &=& UDU^{-1}\\
tr(B) &=& tr(UDU^{-1})\\
tr(B) &=& tr(U^{-1}UD)\\
tr(B) &=& tr(D)\\
tr(B) &=& \sum_i \lambda_i
\end{eqnarray}

\section{Assignment}		%5
A hypreplane or affine set $H \in \mathbb{R}^n$ defined by the equation $h(x) \equiv w_0 + w^Tx = 0$.

To show:
\begin{enumerate}
\item for any point $x_0 \in H, w^Tx_0=-w_0$.
\item if $x_1$ and $x_2$ lie in $H$, then $w^T(x_1-x_2)=0$.
\item $\hat{w}=w/||w||$ is the vector normal to the surface of $H$.
\item if $x_0 \in H$, then the sgned distance of any point $x$ to $H$ is given by $\hat{w}^T(x-x_0) = (w^Tx+w_0)/||w||$
\end{enumerate}  

(1):

\begin{eqnarray}
h(x_0) \equiv w_0 + w^T x_0 &=& 0\\
w^T x_0 &=& -w_0
\end{eqnarray}

(2):

\begin{eqnarray}
h(x_1) \equiv w_0 + w^T x_1 = 0 \label{x1}\\
h(x_2) \equiv w_0 + w^T x_2 = 0 \label{x2}
\end{eqnarray}
With \ref{x1} - \ref{x2}:
\begin{eqnarray}
w_0 + w^Tx_1-w_0-w^Tx_2 = 0\\
w^T(x_1-x_2)=0
\end{eqnarray}

(3):

Let $v \in H$. The scalar product of $v$ and $\hat{w}$ has to be zero since all vectors in H have to be orthonogal to $\hat{w}$. Let $x_1, x_2 \in H$ and represent $v$ as a linear combination of $x_1$ and $x_2$ so that $v = x_1 - x_2$.

\begin{eqnarray}
<\hat{w}, v> = \hat{w}^T v = \hat{w}^T (x_1-x_2) = 0\\
\Leftrightarrow \frac{w^T}{||w||}(x_1-x_2) = 0
\end{eqnarray}

We know that it is equal to zero due to the knowledge of (2).

(4):

Computing the distance $d$ of a point $x$ to $H$ can be computed using the projection so that $d = \frac{<\hat{w},(x-x_0)>}{||\hat{w}||}$. Since $\hat{w}$ is already normalized to a length of 1, the equation can be transformed to $\hat{w}^T (x-x_0)$.
\begin{eqnarray}
\hat{w}^T(x-x_0) &=& \frac{w^Tx+w_0}{||w||}\\
\frac{w^T}{||w||}(x-x_0) &=& \frac{w^Tx+w_0}{||w||}\\
w^T(x-x_0) &=& w^Tx+w_0\\
\end{eqnarray}
with $-w_0 = w^Tx_0$:
\begin{eqnarray}
w^T (x-x_0) &=& w^Tx -w^T x_0\\
w^T (x-x_0) &=& w^T (x-x_0)
\end{eqnarray}

\end{document}
