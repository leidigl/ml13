\documentclass{article}
\usepackage{ml1_homework_template}
\usepackage{amsmath}
\usepackage{amssymb}

% please submit the corresponding pdf by email to
% homework@class,brml.org, and write "homework sheet xx" in the 
% title.  No more, no less!  (Instead of xx, however,
% put the decimal number of the homework sheet.)

% Please update the following line, only change XX to the homework
% sheet number
\title{homework sheet 07}


\author{
\name{Andre Seitz}\\
\imat{03622870}\\
\email{andre.seitz@mytum.de}
\And
\name{Linda Leidig} \\
\imat{03608416}\\
\email{linda.leidig@tum.de}
}

% The \author macro works with any number of authors. There are two commands
% used to separate the names and addresses of multiple authors: \And and \AND.
%
% Using \And between authors leaves it to \LaTeX{} to determine where to break
% the lines. Using \AND forces a linebreak at that point. So, if \LaTeX{}
% puts 3 of 4 authors names on the first line, and the last on the second
% line, try using \AND instead of \And before the third author name.


\renewcommand{\Vec}[1]{\ensuremath{\mathbf{#1}}}
\newcommand{\Mtx}[1]{\ensuremath{\mathbf{#1}}}
\newcommand{\R}{\ensuremath{\mathbb{R}}}


\begin{document}
\maketitle

\section{Assignment: The Gaussian Kernel}
\paragraph*{Problem 1}
$\;$ 

To show: $K(x,y) = exp\left(-\frac{|\Vec{x}-\Vec{y}|^2}{2 \sigma^2}\right)$ is a kernel.

\begin{eqnarray}
K(x,y) &=& exp\left(-\frac{|\Vec{x}-\Vec{y}|^2}{2 \sigma^2}\right)\\
&=& exp\left(-\frac{\Vec{x}^T\Vec{x}-2\Vec{x}^T\Vec{y}+\Vec{y}^T\Vec{y}}{2 \sigma^2}\right)\\
&=& \underbrace{exp\left( -\frac{\Vec{x}^T\Vec{x}}{2 \sigma^2}\right) exp\left( -\frac{\Vec{y}^T\Vec{y}}{2 \sigma^2}\right)}_{\text{1. has to be a kernel}} \underbrace{exp\left( \frac{\Vec{x}^T\Vec{y}}{\sigma^2}\right)}_{\text{2. has to be a kernel}} \label{two_terms}
\end{eqnarray}
If both are kernels the entire term is a kernel (rule 3).

To show that 1. is a kernel we have a look on the linear kernel $K_3(\Vec{x}, \Vec{y}) = \Vec{x}^T\Vec{y}$. With $\phi(z) = exp\left( -\frac{\Vec{z}^T\Vec{z}}{2\sigma^2}\right)$ and rule 4 we have the following kernel:
\begin{eqnarray}
K_3(\phi(\Vec{x}), \phi(\Vec{y})) &=& \phi(\Vec{x})^T\phi(\Vec{y})\\
&=& \left(exp\left( -\frac{\Vec{x}^T\Vec{x}}{2\sigma^2}\right)\right)^T exp\left( -\frac{\Vec{y}^T\Vec{y}}{2\sigma^2}\right)\\
&=&exp\left( -\frac{\Vec{x}^T\Vec{x}}{2\sigma^2}\right) exp\left( -\frac{\Vec{y}^T\Vec{y}}{2\sigma^2}\right)
\end{eqnarray}
The last step is possible since it is a scalar. Therefore, the first part is a kernel.

To show that 2. is a kernel we have a look on the linear kernel $K_3$ again and additionally we use the Taylor expansion. We want to prove that $exp(K_1(\Vec{x}, \Vec{y}))$ is a kernel if $K_1$ is a kernel. With the Taylor expansion we know
\begin{eqnarray}
exp(K_1(\Vec{x}, \Vec{y})) = \sum_{i = 0}^{\infty} \frac{(K_1(\Vec{x}, \Vec{y}))^i}{i!}
\end{eqnarray}
With rule 3 we know that $(K_1(\Vec{x}, \Vec{y}))^i$ is a kernel. From rule 2 with $a = \frac{1}{i!}$ we know that $\frac{(K_1(\Vec{x}, \Vec{y}))^i}{i!}$ is also a kernel. Applying rule 1 we get that $\sum_{i = 0}^{\infty} \frac{(K_1(\Vec{x}, \Vec{y}))^i}{i!}$ is also a kernel which leads to the conclusion that $exp(K_1(\Vec{x}, \Vec{y}))$ is a kernel. Here $K_1(\Vec{x}, \Vec{y}) = \frac{\Vec{x}^T\Vec{y}}{\sigma^2}$. We know that $\Vec{x}^T\Vec{y}$ is a kernel. Applying rule 3 leads to the conclusion that $\frac{\Vec{x}^T\Vec{y}}{\sigma^2}$ is also a kernel. Therefore, the second term of equation \ref{two_terms} is also a kernel.

We could show that both terms of equation \ref{two_terms} are kernels. Therefore, the whole equation is a kernel.


\paragraph*{Problem 2}
$\;$ 

To do: Determine $\phi(\Vec{x})$ so that
\begin{eqnarray}
\phi(\Vec{x})^T\phi(\Vec{y}) = exp(-\frac{|\Vec{x}-\Vec{y}|^2}{2 \sigma^2})
\end{eqnarray}

With the Taylor expansion we get:
\begin{eqnarray}
&&exp(-\frac{|\Vec{x}-\Vec{y}|^2}{2 \sigma^2})\\
&=& exp\left( -\frac{\Vec{x}^T\Vec{x}}{2 \sigma^2}\right) exp\left( -\frac{\Vec{y}^T\Vec{y}}{2 \sigma^2}\right) \sum_{n=0}^{\infty}\frac{\left(\frac{\Vec{x}^T\Vec{y}}{\sigma^{2n}} \right)^n}{n!}\\
&=& exp\left( -\frac{\Vec{x}^T\Vec{x}}{2 \sigma^2}\right) exp\left( -\frac{\Vec{y}^T\Vec{y}}{2 \sigma^2}\right) \sum_{n=0}^{\infty}\frac{\left(\Vec{x}^T\Vec{y} \right)^n}{n!\sigma^{2n}}\\
&=& exp\left( -\frac{\Vec{x}^T\Vec{x}}{2 \sigma^2}\right) exp\left( -\frac{\Vec{y}^T\Vec{y}}{2 \sigma^2}\right) \sum_{n=0}^{\infty}\frac{\left(\sum_{t=1}^m x_t y_t \right)^n}{n!\sigma^{2n}}\\
&=& exp\left( -\frac{\Vec{x}^T\Vec{x}}{2 \sigma^2}\right) exp\left( -\frac{\Vec{y}^T\Vec{y}}{2 \sigma^2}\right) \sum_{n=0}^{\infty}\frac{\sum_{k_1+...+k_m = n} \left(\begin{array}{c}
n\\k_1,...,k_m
\end{array}\right) \prod_{t=1}^m(x_t y_t)^{k_t}}{n!\sigma^{2n}}\\
&=& exp\left( -\frac{\Vec{x}^T\Vec{x}}{2 \sigma^2}\right) exp\left( -\frac{\Vec{y}^T\Vec{y}}{2 \sigma^2}\right) \sum_{n=0}^{\infty}\frac{\sum_{k_1+...+k_m = n} \frac{n!}{k_1!...k_m!} \prod_{t=1}^m(x_t y_t)^{k_t}}{n!\sigma^{2n}}\\
&=& exp\left( -\frac{\Vec{x}^T\Vec{x}}{2 \sigma^2}\right) exp\left( -\frac{\Vec{y}^T\Vec{y}}{2 \sigma^2}\right) \sum_{n=0}^{\infty}\sum_{k_1+...+k_m = n} \frac{1}{k_1!...k_m!\sigma^{2n}} \prod_{t=1}^m x_t^{k_t} \prod_{t=1}^m y_t^{k_t}\\
&=& \sum_{n=0}^{\infty}\sum_{k_1+...+k_m = n} \left[exp\left( -\frac{\Vec{x}^T\Vec{x}}{2 \sigma^2}\right) \frac{\prod_{t=1}^m x_t^{k_t}}{k_1!...k_m!\sigma^{2n}}\right] \left[exp\left( -\frac{\Vec{y}^T\Vec{y}}{2 \sigma^2}\right) \frac{ \prod_{t=1}^m y_t^{k_t}}{k_1!...k_m!\sigma^{2n}}\right]
\end{eqnarray}
The inner product of the last equation is split into $\Vec{x}$ and $\Vec{y}$ and is therefore the inner product of the infinite-dimensional feature space.


\paragraph*{Problem 3}
$\;$ 

\section{Assignment: Kernel Perceptron}
\paragraph*{Problem 4}
$\;$ 

\section{Assignment: Kernelized k-nearest neighbors}
\paragraph*{Problem 5}
$\;$ 

\section{Assignment: Convex functions}
\paragraph*{Problem 6}
$\;$ 

\paragraph*{Problem 7}
$\;$ 

\paragraph*{Problem 8}
$\;$ 

\end{document}
